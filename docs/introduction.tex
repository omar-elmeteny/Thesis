\chapter{Introduction}
\section{Motivation}
Computer programming is of the most significant importance in today's digital age because it plays a significant role in various aspects of our lives and has a remarkable impact on numerous industries and fields. It is essential due to its problem-solving capabilities, automation potential, fostering innovation, offering career opportunities, enhancing digital literacy, promoting computational thinking, enabling interdisciplinary applications, and empowering individuals. As technology advances, programming skills will become increasingly valuable in shaping the future. 

In the Islamic golden age, Arabic played a significant role as the language of sciences. Arab scholars and scientists made significant contributions in various fields, such as mathematics, astronomy, medicine, chemistry, physics, and philosophy. The wide spread of Arabic has faded away, and currently, English is the most prominent language, especially in the field of computer science. This created a barrier for non-English speakers to delve into the field of computer programming. My aim is to remove this barrier.
\section{Problem Statement}
Nowadays, almost all programming languages are in English, so a person must learn English before learning computer programming. This project aims to eliminate the need to learn English because anyone can write programs using their native language. This will be done by localizing the Java Programming Language.
\section{Thesis Organization}
This paper consists of four chapters. Chapter One is the Introduction, chapter two is the Literature Review and is divided into two sections: Background and Previous Work for Programming Language Localization, chapter three covers the Design and Implementation of the system and is divided into five sections: Project Goals, Project Overview, Design Approach, Implementation Details, and Deploying and Packaging the Eclipse Plugin, and finally chapter four is the Conclusion, and it is divided into three sections: Achievements, Limitations of the project, and Future Work which are features that can be added to the project.
