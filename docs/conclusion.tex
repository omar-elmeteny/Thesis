\chapter{Conclusion}
\section{Achievements}
What I achieved in this project is that I created an Eclipse Plugin for Eclipse users to write programs using their own native language (not all languages). Moreover, working on an existing Java source code was achieved by converting it into its localized version. Runtime errors that occurred after running were translated to the user's chosen language. Moreover, the plugin was deployed and packaged for usage on a \ac{PC} or Laptop.
\section{Limitations}
Although I achieved my main goals in this project, I faced some limitations in various aspects that are worth considering:
\subsection{Duplicate Class Names}
One of the limitations of this project is creating two classes, methods, or fields having the same name. This is a problem because a Java (English) identifier can't have multiple translations to any target language. For example, I could have two methods from different classes like Program.run() and Player.run(). In the first case, ``run" means starting the program, while in the second case, ``run" means racing or sprinting. The problem is that the word ``run" will have only one translation when the source code is localized. However, it has two translations in Arabic: Program.run() must be ()\<شغل>\<.>\<برنامج> and Player.run() must be ()\<اجري>\<.>\<لاعب>, but unfortunately, it will have one translation only.
\subsection{Using Third Party Libraries} 
To use third-party libraries, the user must enter the translations of packages, classes, methods, and fields in the translations.guct file directly instead of using the Identifiers Page (discussed in the Eclipse Plugin section) because the tree only represents the built-in Java identifiers.
\subsection{Supporting new Java versions}
New Java versions can't be supported in this project because they will have new grammar rules. Therefore, my lexer and parser cannot deal with those rules. For example, if I try to use a keyword that doesn't exist, the lexer will not be able to tokenize it, and the transpilation process will fail. Moreover, if a new grammar rule is added, it won't be supported. 
\subsection{Supporting all Eclipse versions}
My Eclipse Plugin was done using Eclipse IDE 2022-12 and was tested on the same Eclipse version. However, the plugin wasn't tested on any other Eclipse versions, so it is assumed that my plugin doesn't work except on Eclipse IDE 2022-12.
\subsection{Eclipse and Maven}
When users write a program using my plugin, before running the code, they must build the project and update the Maven project. This is because Eclipse doesn't respect the Maven build lifecycle.
\section{Future Work}
To improve the development experience for the localized programming language, several areas of improvement can be explored. The following aspects are worth considering for future development:
\subsection{Compiler Time Errors}
Any programming language must have the capability of detecting problems during compilation. The error-reporting capabilities of the compiler can be improved in order to strengthen this feature of the localized programming language. You can achieve this by developing a server that implements \ac{LSP} and serves as a proxy between the IDE or code editor and the current Java LSP implementation. In order to translate errors in responses from the Java LSP, the proxy server will intercept requests, and transpile the document on the fly. It will translate the error messages' Java-specific terminology to the vocabulary of the localized programming language. This step makes sure that users get compiler errors in the language they are using straight in their IDE, which makes it easier to comprehend and troubleshoot. In addition, a new Maven Plugin can be implemented that also acts as a proxy for the maven-compiler-plugin to translate compiler error messages before reporting them to the Maven build automation tool. This will ensure that project builds using the command line also display localized compiler error messages.
\subsection{Enhanced IDE Features}
It is crucial to integrate common IDE features into the ecosystem of the localized programming language to increase the productivity of the development process. Currently, no features are added and the source code is treated like text. Features like go-to definition, auto-complete, documentation on hover, and syntax highlighting could be added. These objectives can also be accomplished with the aid of the LSP Server mentioned in the preceding section.
\subsection{Support for Other IDEs}
Expanding the compatibility of the localized programming language with other popular IDEs is another important consideration. Currently, the project is focused on Eclipse, but extending support to other widely used IDEs, such as IntelliJ, Visual Studio Code, Atom, PyCharm, and other IDEs, will enable users to leverage their preferred development environments. 
\subsection{Integration for Other Build Automation Tools}
The localized programming language's adaptability and popularity will be boosted by integrating it with various build automation tools in addition to IDE support. Maven is the only build automation tool that is currently supported. The industry uses well-known build automation systems including Apache, ANT, CMake, Sonatype Nexus, Bazel, and Gradle. By supporting them, users can use the build automation tool they are comfortable with the most. 
\subsection{Support for Other Programming Languages}
Although the project initially concentrates on supporting Java, it is important to take into consideration the prospect of extending support to other programming languages. As a result, different techniques for integration may be required for each language. But similar ideas and strategies for creating the localized programming language can be used to support high-level programming languages like C, C++, and others. The most comfortable programming language for each user will be made available by this expansion.
\subsection{Debugging Support}
The localized programming language should include effective debugging features to help users find and fix problems. This can be done by developing a debug server implementing the \ac{JDI} that proxies an existing implementation of JDI in a similar manner to the LSP proxy discussed earlier. This will allow users to add breakpoints in code written in the localized language, step through the code and view variable names in their original language. 
\section{Summary}
In this project, I created an Eclipse Plugin to make it easier for non-English speaking programmers, especially children, to write computer programs using their native language. This was done by localizing the Java Language. A user interface was created for the user to be able to translate all Java keywords, identifiers, and exceptions. The source code the user wrote was transpiled to its equivalent Java file and then compiled, and runtime errors were returned to the user in his language. To create the plugin with localization features, other components helped in creating it, like the Translator, Transpiler, and Maven Plugin. 


